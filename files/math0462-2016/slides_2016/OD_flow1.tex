%% $Id: louveaux-epfl06.tex,v 1.5 2006/07/10 13:46:20 louveaux Exp $
%%\documentclass[9pt,trans]{beamer}
\documentclass[9pt]{beamer}
\usepackage{beamerfoils}%% FoilTeX emulation
\usepackage{epsfig}
\usepackage{eurosym}
\mode<presentation>
{
  \usetheme{Boadilla}
  % oder ...

  \setbeamercovered{transparent}
  % oder auch nicht
}
\usepackage[french]{babel}
\usepackage[latin1]{inputenc}
%%\usepackage{times}
%%\usepackage[T1]{fontenc}
%\usepackage{booktabs}

%%\includeonlyframes{current}

\title{Discrete Optimization}

\author{Quentin
Louveaux}
\institute{ULg - Institut Montefiore}
\date{2016}

% Falls eine Logodatei namens "university-logo-filename.xxx" vorhanden
% ist, wobei xxx ein von latex bzw. pdflatex lesbares Graphikformat
% ist, so kann man wie folgt ein Logo einf|gen:

% \pgfdeclareimage[height=0.5cm]{university-logo}{university-logo-filename}
% \logo{\pgfuseimage{university-logo}}

% Folgendes sollte gelvscht werden, wenn man nicht am Anfang jedes
% Unterabschnitts die Gliederung nochmal sehen mvchte.
%% \AtBeginSection[]
%% {
%%   \begin{frame}<beamer>
%%     \frametitle{Gliederung}
%%     \tableofcontents[currentsection,currentsubsection]
%%   \end{frame}
%% }

% Falls Aufzdhlungen immer schrittweise gezeigt werden sollen, kann
% folgendes Kommando benutzt werden:

\beamerdefaultoverlayspecification{<+->}

%%%%%%
\definecolor{rot}{rgb}{1,0,0}
\definecolor{gruen}{rgb}{0,1,0}
\definecolor{blau}{rgb}{0,0,1}

%%% number sets
\newcommand{\Z}       {\mathbb{Z} }
\newcommand{\R}       {\mathbb{R} }
\newcommand{\Q}       {\mathbb{Q} }
\newcommand{\N}       {\mathbb{N} }
\newcommand{\spa}     {\text{span}}
\newcommand{\lin}     {\text{span}}
\newcommand{\inter}   {\text{int} }


%%% mathematical stuff
\newcommand{\sosR}    {\sum^2}
\newcommand*{\transpose}[1]  { {#1}^T }
\newcommand*{\rounddown}[1]  {\left\floor #1 \right\rfloor}
\newcommand*{\roundup}[1]    {\left\lceil  #1 \right\rceil}
\newcommand*{\ipart}[1]      {\rounddown{#1}}
\newcommand*{\fpart}[1]      {\mathfrak{f}\left(#1\right)}


\newcommand*{\iepoly}[2]  {z_{#1}\left(#2\right)}
\newcommand*{\redmon}[3]  {r_{#1}^{#2}\left( #3 \right)}
\newcommand*{\redset}[1]  {#1^{\emph{red}}}

\newcommand*{\Gpoly}[1] {P_{[#1]}}

\newcommand*{\nonc}[1]{\overline{#1}}
\newcommand*{\const}[1]{#1_0}

\newcommand{\define}{\stackrel{\rm def.}{\Leftrightarrow}}
\newcommand{\qform}[3]{\frac{1}{2} x^{\top}#1x + #2^{\top}x + #3}
\def\xzero{x^{0}}
\newcommand{\gxh}[2]{{g_{#1}(#2)}}

\def\pcone_k{{\mathcal C}_{k}(f)}
\def\orthant_j{{\mathcal O}_{j}}

\def\BDB{BDB^{\top}}
\def\LDL{LDL^{\top}}
\def\bA{A}
\def\bb{b}
\def\bc{c}
\def\bh{h}
\def\bp{p}
\def\bx{x}
\def\by{y}
\def\bu{u}
\def\bv{v}
\def\bd{d}
\def\T{^{T}}
\def\D{}
\def\mb{{\bf}}
\def\sep{}
\def\bo{0}
\def\bw{w}
\def\ba{a}
\def\bg{g}
\def\bH{H}
\def\be{e}

\let\ve=\relax
\newcommand\vealpha{{\alpha}}
\newcommand{\st}{\mathrm{s.t.}}
\DeclareMathOperator\conv{conv}
\DeclareMathOperator\cone{cone}
\newcommand{\B}{\{0,1\}}

\newcommand*{\person}[1] {\textsc{#1}}

\newtheorem{algorithm}{Algorithmus}

\makeatletter
\newenvironment{rmat}{\left(\null\,\vcenter\bgroup
  \Let@\restore@math@cr\default@tag
  \baselineskip6\ex@ \lineskip1.5\ex@ \lineskiplimit\lineskip
  \ialign\bgroup\hfil$\m@th\scriptstyle##$&&\thickspace
  \hfil$\m@th\scriptstyle##$\crcr
}{%
  \crcr\egroup\egroup\,\right)%
}
\makeatother

%%%%%%%%%%%%%%%%%%%%%%%%%%%%%%%%%%%%%%%%%%%%%%%%%%%%%%%%%%%%%%%%%%%%%%%%%%%%%%%%
%\begin{frame}
  %\titlepage
%\end{frame}

%% \begin{frame}
%%   \frametitle{Gliederung}
%%   \tableofcontents
%%   % Die Option [pausesections] kvnnte n|tzlich sein.
%% \end{frame}


%%%%%%%%%%%%%%%%%%%%%%%%%%%%%%%%%%%%%%%%%%%%%%%%%%%%%%%%%%%%%%%%%%

\definecolor{orange}{rgb}{0.8,0.3,0.0}
\definecolor{darkgreen}{rgb}{0.0,0.5,0.0}
\definecolor{gold}{rgb}{1.0,0.8,0.0}
\definecolor{brown}{rgb}{0.6,0.2,0.2}
\definecolor{blue4}{rgb}{0,0,144}
\definecolor{white}{rgb}{255,255,255}
\definecolor{blueexample}{rgb}{0.2,0.2,0.7}
\begin{document}
\begin{frame}
  \titlepage
\end{frame}
\begin{frame}
\frametitle{The maximum flow problem}
Let $G=(V,E)$ be a \alert{directed graph} with a \alert{source node} $s$
and a \alert{sink node} $t$ and with \alert{capacities} $u_j$ on each edge $e_j.$
\begin{block}{The maximum flow problem}
Find the maximum flow sent
from $s$ to $t$ where the flow \alert{entering} each node (except $s$ and $t$)
is equal to the flow \alert{leaving} each node and where the flow
on \alert{each edge} satisfies the capacity.
\end{block}
\end{frame}
\begin{frame}
\frametitle{The max flow problem}
Send the maximum flow from a node $s$ (\alert{source}) to  a node $t$ (\alert{sink}).
\uncover<2->{\\Each \alert{edge} has a \alert{capacity}}
\uncover<3>{\\\textcolor{darkgreen}{Example} of a feasible flow}
\begin{overlayarea}{\linewidth}{6cm}
\begin{center}
\includegraphics<1>[width=8cm]{stflot.pdf}
\includegraphics<2>[width=8cm]{stflotcapacite.pdf}
\includegraphics<3>[width=8cm]{stflotFlot.pdf}
\end{center}
\end{overlayarea}
\end{frame}
\begin{frame}
\frametitle{The minimum cut problem}
\begin{block}{The minimum $s-t$-cut problem}
An $s-t$-cut is a partition of the nodes of $G$ into $V_1$ and $V_2$ where
$s\in V_1$ and $t\in V_2$. The \alert{value} of an $s-t$-cut is equal to
the \alert{sum of the capacities} of all edges joining one node of $V_1$
to one node of $V_2$.\\
The \alert{min-cut problem} is the problem of finding the $s-t$-cut
that has the minimal value.
\end{block}
\begin{block}{Important result}
Max flow = min cut
\end{block}
\end{frame}
\begin{frame}
\frametitle{Examples of applications}
\begin{block}{The problem of representatives}
A town has $r$ \alert{residents} $R_1,\ldots, R_r$,
$q$ \alert{clubs} $C_1,\ldots, C_q$ and $p$
\alert{political parties} $P_1,\ldots,P_p.$
Each resident is member of \alert{at least} one club and
can belong to \alert{exactly one} political party.\bigskip

Each club must nominate one of its members to represent it
in the town's council in such a way that the number of council members
belonging to the political party $P_k$ is \alert{at most} $u_k$.
\end{block}
Problem with applications in \alert{resource assignement settings}.
\end{frame}
\begin{frame}
\frametitle{Examples of applications}
\begin{block}{Matrix rounding problem}
We are given a \alert{matrix} $D\in \mathbb R^{p\times q}$
with row sums $\alpha_i$ and column sums $\beta_j$.
We can decide to round any element in the matrix to the \alert{lower}
or the \alert{upper} integer.\bigskip

How can we round each element such that, for each row
and each column,  the sum of the \alert{rounded elements}
is equal to a \alert{rounding of the sums}.\bigskip

This is called a \alert{consistent rounding}.
\end{block}
\end{frame}
\begin{frame}
\frametitle{Examples of applications}
\begin{block}{Scheduling on uniform parallel machines}
We want to schedule $J$ jobs on $M$ parallel machines.\\
Each job $j$ has a \alert{processing time} $p_j$,
a \alert{release date} $r_j$ and a \alert{due date } $d_j$.
Furthermore we allow \alert{preemption}.
\end{block}
\end{frame}
\begin{frame}
\frametitle{Examples of applications for the min-cut problem}
\begin{block}{Distributed computing on a 2-processor computer}
Consider a \alert{large computer program} that has to be run
on \alert{two processors}. The code is subdivided in \alert{tasks}
that have a different processing time on both processors.

Furthermore there is a \alert{communication cost} if some
tasks are performed on \alert{different processors}.
\end{block}
\end{frame}
\begin{frame}
\frametitle{Example of application of the min-cut problem}
\begin{block}{Segmentation of an image}
Consider a picture taken of of an object: for example, an MRI image of a heart.
We may want to detect, for each pixel, whether it actually belongs to
the heart or whether it belongs to the background.\bigskip

It is possible to create a \alert{similarity function} for each pair of pixel
that is \alert{large} if the 2 pixels are likely to belong to the same part of the picture.
\end{block}
\end{frame}
\begin{frame}
\frametitle{Flows and cuts}
\begin{block}{The residual network}
Given a flow $x$, the \alert{residual capacity} $r_{ij}$ is
the maximum additional flow that can be \alert{added} to arc $(i,j)$
or \alert{removed} from arc $(j,i)$.
$$ r_{ij} = (u_{ij}-x_{ij}) + x_{ji}$$
This yields the graph $G(x)$, the \alert{residual network}.
\end{block}
\begin{overlayarea}{\linewidth}{4.9cm}
\begin{center}
\includegraphics<2>[height=4.9cm]{stflotFlot2.pdf}
\includegraphics<3>[height=4.9cm]{residual1.pdf}
\includegraphics<4>[height=4.9cm]{residual2.pdf}
\includegraphics<5>[height=4.9cm]{residualfinal.pdf}
\end{center}
\end{overlayarea}
\end{frame}
\begin{frame}
\frametitle{$(s-t)-$cuts}
\begin{block}{Definition of an $(s-t)-$cut}
A cut is a partition of the \alert{node set} into two
subsets $S,\bar S$ such that
\begin{itemize}
\item<1-> $S\cup \bar S = V$
\item<1-> $S\cap \bar S=\emptyset$
\item<1-> It is an \alert{$(s-t)-$cut} if $s\in S$ and $t\in \bar S$.
\end{itemize}
\end{block}
\uncover<2->{The \alert{ capacity} of an $(s-t)-$cut is the sum of the capacities of the
\alert{forward arcs} of the cut:
$$u(S,\bar S) = \sum_{(i,j)\in (S,\bar S)} u_{ij}.$$}
\uncover<3->{The \alert{residual capacity} of an $(s-t)-$cut 
is the sum of the residual capacities of the
\alert{forward arcs} of the cut:
$$u(S,\bar S) = \sum_{(i,j)\in (S,\bar S)} r_{ij}.$$}
\uncover<4->{\alert{Property:} The value of any flow is less or equal to
the capacity of \alert{any cut} in the network.\\
This is a \alert{weak duality } property.}
\end{frame}
\begin{frame}
\frametitle{Generic augmenting path algorithm}
\begin{block}{Definition}
A \alert{directed path} from the source $s$ to the sink $t$ in the
residual network is an \alert{augmenting path}.\\
The residual capacity of the augmenting path is the 
\alert{minimum} residual capacity over all arcs in the path.
\end{block}
Idea: augment step by step the value of the current flow
using paths in the residual network.\bigskip

\uncover<2->{The value by which we can augment the flow is less or equal
to the \alert{residual capacity} of any cut.}
\end{frame}
\begin{frame}
\frametitle{Generic augmenting path algorithm}
\texttt{x:=0};\\
\texttt{while} $G(x)$ contains a directed path from $s$ to $t$ \texttt{do}\\
	$\qquad$ Find an augmenting path $P$\\
	$\qquad \delta:= \min\{r_{ij} \mid (i,j)\in P\}$\\
	$\qquad$ Augment $\delta$ units of flow along $P$ and update $G(x)$
\bigskip

\uncover<2->{Important questions:\\
Does the algorithm terminate?\\
Does it provide an optimal solution?\\
What is its running time?}
\end{frame}
\begin{frame}
\frametitle{Labeling algorithm}
This is a first implementation of the augmenting path algorithm
that \alert{does not run in polynomial time}.\bigskip

\uncover<2->{\alert{Idea}: Perform a breadth-first-search or a 
depth-first-search on the graph to \alert{label} the nodes.
If the sink node is labeled, we have found an \alert{augmenting path}.}

\uncover<3->{\begin{block}{The algorithm is correct}
If, at an iteration, the sink node \alert{cannot be labeled},
then the \alert{flow is maximal}.\end{block}}
\end{frame}
\begin{frame}
\frametitle{The max flow-min cut theorem}
\begin{block}{Theorem}
The maximum value of of the flow from $s$ to $t$ equals the minimum
capacity among all $(s-t)-$cuts.
\end{block}
\begin{block}{Augmenting path theorem}
A flow $x^*$ is a \alert{maximum flow}
if and only if the residual network $G(x^*)$
\alert{contains no augmenting path}.
\end{block}
\begin{block}{Integrality theorem}
If all arc capacities are \alert{integer}, the maximum flow problem
has an \alert{integer} maximum flow.
\end{block}
\end{frame}
\begin{frame}
\frametitle{Complexity of the labeling algorithm}
\begin{block}{Theorem}
The labeling algorithm solves the maximum flow problem
in $\mathcal O(mnU)$ operations.
\end{block}
This implies that in a such an implementation, the algorithm
\alert{does not run } in polynomial time in the \alert{worst case}.
\end{frame}
\begin{frame}
\frametitle{Combinatorial implications}
\begin{itemize}
\item<1-> The maximum number of \alert{arc-disjoint paths} from $s$ to $t$
is equal to the minimum number of arcs whose \alert{removal disconnects} $s$ from $t$.
\item<1-> The maximum number of \alert{node-disjoint} paths from $s$ to $t$ is
equal to the minimum number of \alert{nodes whose removal} disconnects $s$ from $t$.
\item<1-> Matchings and covers are strong dual for \alert{bipartite graphs}.
\end{itemize}
\end{frame}
\begin{frame}
\frametitle{Flows with lower bounds}
\begin{block}{Optimizing from a feasible flow}
Just change the computation of the \alert{residual capacity}
$$ r_{ij} = (u_{ij}-x_{ij}) + (x_{ji}-l_{ji})$$
\end{block}
\begin{block}{Establishing a feasible flow}
\begin{itemize}
\item<2-> Add an arc $(t,s)$ of \alert{infinite capacity}
and transform the problem into finding a \alert{feasible circulation}.
\item<2-> At each node, compute 
$$b(i) = \sum_{j\mid (j,i)\in A} l_{ji} - \sum_{j\mid (i,j)\in A} l_{ij}$$
\item<2-> Put only \alert{upper bounds} on the arcs to find $x'$
\item<2-> Create a \alert{dummy source node} $s'$ and a \alert{dummy sink node} $t'$.
\item<2-> Create an edge from $s'$ to $i$ if $b(i)>0$ with capacity $b(i)$.
\item<2-> Create an edge from $i$ to $t'$ if $b(i)<0$ with capacity $|b(i)|$.
\item<2-> The problem is feasible if there exists a flow with total value $\sum_ib(i)$
\item<2-> The potential feasible flow is obtained as $x_{ij}:=x_{ij}'+l_{ij}$
\end{itemize}
\end{block}
\end{frame}
\end{document}
