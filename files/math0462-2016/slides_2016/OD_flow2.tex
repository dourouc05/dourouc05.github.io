%% $Id: louveaux-epfl06.tex,v 1.5 2006/07/10 13:46:20 louveaux Exp $
%%\documentclass[9pt,trans]{beamer}
\documentclass[9pt]{beamer}
\usepackage{beamerfoils}%% FoilTeX emulation
\usepackage{epsfig}
\usepackage{eurosym}
\mode<presentation>
{
  \usetheme{Boadilla}
  % oder ...

  \setbeamercovered{transparent}
  % oder auch nicht
}
\usepackage[french]{babel}
\usepackage[latin1]{inputenc}
%%\usepackage{times}
%%\usepackage[T1]{fontenc}
%\usepackage{booktabs}

%%\includeonlyframes{current}

\title{Discrete Optimization}

\author{Quentin
Louveaux}
\institute{ULg - Institut Montefiore}
\date{2016}

% Falls eine Logodatei namens "university-logo-filename.xxx" vorhanden
% ist, wobei xxx ein von latex bzw. pdflatex lesbares Graphikformat
% ist, so kann man wie folgt ein Logo einf|gen:

% \pgfdeclareimage[height=0.5cm]{university-logo}{university-logo-filename}
% \logo{\pgfuseimage{university-logo}}

% Folgendes sollte gelvscht werden, wenn man nicht am Anfang jedes
% Unterabschnitts die Gliederung nochmal sehen mvchte.
%% \AtBeginSection[]
%% {
%%   \begin{frame}<beamer>
%%     \frametitle{Gliederung}
%%     \tableofcontents[currentsection,currentsubsection]
%%   \end{frame}
%% }

% Falls Aufzdhlungen immer schrittweise gezeigt werden sollen, kann
% folgendes Kommando benutzt werden:

\beamerdefaultoverlayspecification{<+->}

%%%%%%
\definecolor{rot}{rgb}{1,0,0}
\definecolor{gruen}{rgb}{0,1,0}
\definecolor{blau}{rgb}{0,0,1}

%%% number sets
\newcommand{\Z}       {\mathbb{Z} }
\newcommand{\R}       {\mathbb{R} }
\newcommand{\Q}       {\mathbb{Q} }
\newcommand{\N}       {\mathbb{N} }
\newcommand{\spa}     {\text{span}}
\newcommand{\lin}     {\text{span}}
\newcommand{\inter}   {\text{int} }


%%% mathematical stuff
\newcommand{\sosR}    {\sum^2}
\newcommand*{\transpose}[1]  { {#1}^T }
\newcommand*{\rounddown}[1]  {\left\floor #1 \right\rfloor}
\newcommand*{\roundup}[1]    {\left\lceil  #1 \right\rceil}
\newcommand*{\ipart}[1]      {\rounddown{#1}}
\newcommand*{\fpart}[1]      {\mathfrak{f}\left(#1\right)}


\newcommand*{\iepoly}[2]  {z_{#1}\left(#2\right)}
\newcommand*{\redmon}[3]  {r_{#1}^{#2}\left( #3 \right)}
\newcommand*{\redset}[1]  {#1^{\emph{red}}}

\newcommand*{\Gpoly}[1] {P_{[#1]}}

\newcommand*{\nonc}[1]{\overline{#1}}
\newcommand*{\const}[1]{#1_0}

\newcommand{\define}{\stackrel{\rm def.}{\Leftrightarrow}}
\newcommand{\qform}[3]{\frac{1}{2} x^{\top}#1x + #2^{\top}x + #3}
\def\xzero{x^{0}}
\newcommand{\gxh}[2]{{g_{#1}(#2)}}

\def\pcone_k{{\mathcal C}_{k}(f)}
\def\orthant_j{{\mathcal O}_{j}}

\def\BDB{BDB^{\top}}
\def\LDL{LDL^{\top}}
\def\bA{A}
\def\bb{b}
\def\bc{c}
\def\bh{h}
\def\bp{p}
\def\bx{x}
\def\by{y}
\def\bu{u}
\def\bv{v}
\def\bd{d}
\def\T{^{T}}
\def\D{}
\def\mb{{\bf}}
\def\sep{}
\def\bo{0}
\def\bw{w}
\def\ba{a}
\def\bg{g}
\def\bH{H}
\def\be{e}

\let\ve=\relax
\newcommand\vealpha{{\alpha}}
\newcommand{\st}{\mathrm{s.t.}}
\DeclareMathOperator\conv{conv}
\DeclareMathOperator\cone{cone}
\newcommand{\B}{\{0,1\}}

\newcommand*{\person}[1] {\textsc{#1}}

\newtheorem{algorithm}{Algorithmus}

\makeatletter
\newenvironment{rmat}{\left(\null\,\vcenter\bgroup
  \Let@\restore@math@cr\default@tag
  \baselineskip6\ex@ \lineskip1.5\ex@ \lineskiplimit\lineskip
  \ialign\bgroup\hfil$\m@th\scriptstyle##$&&\thickspace
  \hfil$\m@th\scriptstyle##$\crcr
}{%
  \crcr\egroup\egroup\,\right)%
}
\makeatother

%%%%%%%%%%%%%%%%%%%%%%%%%%%%%%%%%%%%%%%%%%%%%%%%%%%%%%%%%%%%%%%%%%%%%%%%%%%%%%%%
%\begin{frame}
  %\titlepage
%\end{frame}

%% \begin{frame}
%%   \frametitle{Gliederung}
%%   \tableofcontents
%%   % Die Option [pausesections] kvnnte n|tzlich sein.
%% \end{frame}


%%%%%%%%%%%%%%%%%%%%%%%%%%%%%%%%%%%%%%%%%%%%%%%%%%%%%%%%%%%%%%%%%%

\definecolor{orange}{rgb}{0.8,0.3,0.0}
\definecolor{darkgreen}{rgb}{0.0,0.5,0.0}
\definecolor{gold}{rgb}{1.0,0.8,0.0}
\definecolor{brown}{rgb}{0.6,0.2,0.2}
\definecolor{blue4}{rgb}{0,0,144}
\definecolor{white}{rgb}{255,255,255}
\definecolor{blueexample}{rgb}{0.2,0.2,0.7}
\begin{document}
\begin{frame}
  \titlepage
\end{frame}
\begin{frame}
\frametitle{Polynomial running time algorithms for the max flow problem}
\begin{itemize}
\item<1-> The generic augmenting path algorithm runs in $\mathcal O(mnU)$\bigskip
\item<1-> Ways to improve the algorithm and make it run in polynomial time\medskip
\begin{itemize}
\item<1-> Augmenting in \alert{large } increments of flows\\
$\rightarrow$ \alert{Capacity scaling} algorithm\medskip
\item<1-> Augment on \alert{shortest paths} in the residual network\medskip
\item<1-> Relax the \alert{mass balance constraints} and only augment \alert{locally}\\
$\rightarrow$ The \alert{push-relabel} algorithm which is the most efficient one!
\end{itemize}
\end{itemize}
\uncover<2->{In order to implement the last two algorithms, we need to rely
on \alert{distance labels}.}
\end{frame}
\begin{frame}
\frametitle{Distance labels}
\begin{block}{Definition}
A distance function gives a \alert{numeric label} to each node.\\
The distance function is \alert{valid} if
\begin{align*}
d(t)&=0\\
d(i)&\leq d(j)+1\quad \text{for every arc }(i,j)\text{ in } G(x)
\end{align*}
\end{block}
\begin{block}{Property}
If $d$ is valid, $d(i)$ is a \alert{lower bound} on the length
of the shortest path from $i$ to $t$ in the \alert{residual graph}.
\end{block}
\begin{block}{Corollary}
If $d(s)\geq n$, there exists no path from $s$ to $t$
in the residual graph.
\end{block}
\end{frame}
\begin{frame}
\frametitle{Distance labels}
The distance labels are \alert{exact} if each label
indicates the \alert{exact length} of the shortest path to $t$
in the residual graph.\medskip

\noindent
An exact labeling can be determined in $\mathcal O(m)$ by \alert{backward breadth-first
search}.\medskip

\uncover<2->{An arc $(i,j)\in G(x)$ is \alert{admissible} if $$d(i)=d(j)+1,$$
otherwise it is \alert{inadmissible}.\medskip

An \alert{admissible path} is a path consisting only of admissible arcs.\medskip}

\uncover<3->{An admissible path is a \alert{shortest augmenting path}!}
\end{frame}
\begin{frame}
\frametitle{The shortest augmenting path algorithm}
\begin{itemize}
\item<1-> Augmenting on shortest paths in the residual network guarantees
a \alert{polynomial running time}!
\item<1-> We could rerun backward breadth-first search at \alert{each iteration}
$\rightarrow$ very inefficient but runs in $\mathcal O(nm^2)$
\item<1-> The minimum distance from each node to the sink is 
\alert{monotonically increasing}. 
\end{itemize}
\end{frame}
\begin{frame}
\frametitle{The shortest augmenting path algorithm}
\begin{itemize}
\item Perform an \alert{initial labeling} by backward breadth-first-search
\item Repeat: Perform an \alert{advance operation} (finding an admissible arc from the 
current last node in the admissible path)
\item If we find an augmenting path, then we augment along it!
\item If at some node $i$, we do not find any admissible arc, we \alert{relabel} the node
$$\min\{d(j)+1\mid (i,j)\in \delta(\{i\})\text{ and } r{ij}>0\}$$
and remove the node $i$ from the current admissible path.
\end{itemize}
\end{frame}
\begin{frame}
\frametitle{Correctness of the algorithm}
\begin{itemize}
\item Every operation maintains a \alert{valid distance labeling}.\bigskip
\item Each relabel strictly increases the distance label of a node.\bigskip
\item The shortest augmenting path algorithm correctly computes a mximum flow.
\end{itemize}
\end{frame}
\begin{frame}
\frametitle{Complexity of the algorithm}
\begin{itemize}
\item Between two \alert{relabels}, if an arc becomes inadmissible, 
it stays inadmissible.\bigskip
\item If the algorithm relabels any node at most $k$ times, the complexity
of finding admissible arcs and relabeling the nodes is $\mathcal O(km)$
and the algorithm saturates arcs at most $\frac{km}{2}$ times.\bigskip
\item Each distance label increases at most $n$ times. The total number of relabeling
is $n^2$ and the total number of augment operations is $\frac{nm}{2}$.
\end{itemize}
\uncover<4->{The shortest augmenting path algorithm runs
in
$\mathcal O(n^2m)$.}
\end{frame}
\begin{frame}
\frametitle{Preflow-push algorithm}
Drawback of the augmenting path algorithm: the expensive operation
of \alert{sending flow} along a path.\bigskip

Many of these augmentations may share \alert{the same subpath}!\bigskip

Idea of preflow-push: augment along \alert{arcs} and
we therefore have to \alert{relax} the \alert{flow balance constraints}.
\end{frame}
\begin{frame}
\frametitle{Preflows}
\begin{block}{Definition}
A \alert{preflow} is a function $x:A\rightarrow \mathbb R_+$ such that
$$\sum_{j\mid (j,i)\in A} x_{ji} -\sum_{j\mid (i,j)\in A} x_{ij} \geq 0.
\quad \text{for all } i\in V\setminus \{s,t\}.$$
\end{block}
\begin{block}{Definition}
The \alert{excess} at a node $i$ of a given preflow $x$ is given by 
$$e(i):=\sum_{j\mid (j,i)\in A} x_{ji} -\sum_{j\mid (i,j)\in A} x_{ij}.$$
\end{block}
\uncover<3->{A node with positive excess is said to be \alert{active}
because we need to recover the mass-balance constraint at some point.}
\end{frame}
\begin{frame}
\frametitle{Preflow-push algorithm}
\begin{itemize}
\item Compute exact \alert{distance labels} $d(i)$\bigskip
\item Push flows along all arcs emanating from $s$: $x_{sj}:=u_{sj}$\bigskip
\item $d(s):=n$\bigskip
\item If a node is active, \alert{push-relabel}\medskip
\begin{itemize}
\item If $(i,j)$ is admissible, push $\delta:=\min\{e(i),r_{ij}\}$\medskip
\item Else $d(i):=\min\{d(j)+1\mid (i,j)\in \delta(\{i\})\text{ and }r_{ij}>0\}$
\end{itemize}
\end{itemize}
\end{frame}
\begin{frame}
\frametitle{Complexity of the algorithm}
\begin{itemize}
\item Distance labels are always \alert{valid} during the algorithm\medskip
\item At any stage, a node $i$ with excess $e(i)>0$ has a path from $i$ to $s$ in the residual network\medskip
\item For each $i$, \alert{$d(i)<2n$}\medskip
\item The total number of \alert{relabels} is $2n^2$\medskip
\item The algorithm performs at most $nm$ \alert{saturating} pushes.\medskip
\item The algorithm performs at most $n^2m$ \alert{nonsaturating} pushes\bigskip
\item The algorithm runs in $\mathcal O(n^2m)$
\end{itemize}
\end{frame}
\end{document}
