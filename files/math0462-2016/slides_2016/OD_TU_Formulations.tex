%% $Id: louveaux-epfl06.tex,v 1.5 2006/07/10 13:46:20 louveaux Exp $
%%\documentclass[9pt,trans]{beamer}
\documentclass[10pt,handout]{beamer}
\usepackage{beamerfoils}%% FoilTeX emulation
\usepackage{epsfig}
\usepackage{eurosym}
\mode<presentation>

  \usetheme{Boadilla}

  \setbeamercovered{transparent}
 
\usepackage[french]{babel}
\usepackage[latin1]{inputenc}
\title{Discrete Optimization}

\author{Quentin
Louveaux}
\institute{ULg - Institut Montefiore}
\date{2016}

\definecolor{orange}{rgb}{0.8,0.3,0.0}
\definecolor{darkgreen}{rgb}{0.0,0.5,0.0}
\definecolor{gold}{rgb}{1.0,0.8,0.0}
\definecolor{brown}{rgb}{0.6,0.2,0.2}
\definecolor{blue4}{rgb}{0,0,144}
\definecolor{white}{rgb}{255,255,255}
\definecolor{blueexample}{rgb}{0.2,0.2,0.7}
\newcommand{\Z}{\mathbb Z}
\newcommand{\R}{\mathbb R}
\begin{document}
\begin{frame}
  \titlepage
\end{frame}

\begin{frame}
\frametitle{Polynomially solvable problems}
Can we study some types of problems where the linear relaxation \alert{always 
solves} the \alert{discrete} problem.
\begin{block}{Cramer rule}
A solution of $Ax=b$ is given by $x$ with
$$x_i = \frac{\text{det}(A^i)}{\text{det}(A)},$$
with $A^i$ is obtained by taking $A$ and replacing its $i^{th}$ column by $b$.
\end{block}
A polyhedron has all integer vertices if all determinant of all square submatrices
is equal to 1, 0, or -1.
\end{frame}
\begin{frame}
\frametitle{Total unimodularity}
\begin{block}{Definition}
\begin{itemize}
\item<1-> A matrix $A\in \Z^{m\times n}$ is \alert{unimodular} if the determinant
of \alert{each basis} is \alert{1} or \alert{$-1$}.
\item<1-> A matrix $A\in \Z^{m\times n}$ is \alert{totally unimodular} (TU) if the
determinant of each square submatrix of $A$ is \alert{$0,1,-1$}.
\end{itemize}
\end{block}
\begin{block}{Proposition}
\begin{itemize}
\item<2-> A matrix $A$ is TU iff $[A,I]$ is unimodular.
\item<2-> A matrix $A$ is TU iff $\left( \begin{array}{c}A\\-A\\I\\-I\end{array}\right)$ is TU
\item<2-> A matrix $A$ is TU iff $A^T$ is TU
\end{itemize}
\end{block}
\end{frame}
\begin{frame}
\frametitle{Total unimodularity in integer problems}
\begin{block}{Theorem}
\begin{itemize}
\item<1-> $A$ is unimodular iff $P(b)=\{x\in \R^n_+\mid Ax=b\}$
is integral for all $b\in \Z^m$.
\item<1-> $A$ is TU iff $P(b)=\{x\in \R^n_+\mid Ax\leq b\}$ is integral for all $b\in \Z^m.$
\end{itemize}
\end{block}
\begin{block}{Corollary}
\begin{itemize}
\item<2-> $A$ is TU iff $\{x\mid Ax=b, 0\leq x\leq u\}$ is integral for all integral vectors $b$ and $u$.
\item<2-> $A$ is TU iff $\{x\mid a\leq Ax\leq b, l\leq x\leq u\}$ is integral
for all integral vectors $a,b,l,u$.
\end{itemize}
\end{block}
\end{frame}
\begin{frame}
\frametitle{Recognizing total unimodularity}
\begin{block}{Proposition}
A matrix $A$ is TU iff each collection $Q$ of rows  of $A$ can be partitioned into two
parts so that the sum of the rows in one part minus the sum of the rows in the other
part is a vector with entries $0, +1, -1$.
\end{block}
\begin{block}{Theorem}
The following matrices are TU
\begin{itemize}
\item<2-> The node-arc indicence matrix of a \alert{directed graph}
\item<2-> The node-edge incidence matrix of an \alert{undirected bipartite graph}
\item<2-> A $\{0,1\}$-matrix in which each column has its ones consecutively
(also known as \alert{interval matrix})
\end{itemize}
\end{block}
\end{frame}
\begin{frame}
\frametitle{Example of discrete problems solvable with the LP}
\begin{itemize}
\item<1-> The maximum flow problem
\item<1-> The shortest path problem
\item<1-> The minimum cost flow problem
\item<1-> The matching problem in a bipartite graph
\end{itemize}
\end{frame}
\end{document}
