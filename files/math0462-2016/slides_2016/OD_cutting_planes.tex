%% $Id: louveaux-epfl06.tex,v 1.5 2006/07/10 13:46:20 louveaux Exp $
%%\documentclass[9pt,trans]{beamer}
\documentclass[9pt,handout]{beamer}
\usepackage{beamerfoils}%% FoilTeX emulation
\usepackage{epsfig}
\usepackage{eurosym}
\mode<presentation>
{
  \usetheme{Boadilla}
  % oder ...

  \setbeamercovered{transparent}
  % oder auch nicht
}
\usepackage[french]{babel}
\usepackage[latin1]{inputenc}
%%\usepackage{times}
%%\usepackage[T1]{fontenc}
%\usepackage{booktabs}

%%\includeonlyframes{current}

\title{Discrete Optimization}

\author{Quentin
Louveaux}
\institute{ULg - Institut Montefiore}
\date{2016}

% Falls eine Logodatei namens "university-logo-filename.xxx" vorhanden
% ist, wobei xxx ein von latex bzw. pdflatex lesbares Graphikformat
% ist, so kann man wie folgt ein Logo einf|gen:

% \pgfdeclareimage[height=0.5cm]{university-logo}{university-logo-filename}
% \logo{\pgfuseimage{university-logo}}

% Folgendes sollte gelvscht werden, wenn man nicht am Anfang jedes
% Unterabschnitts die Gliederung nochmal sehen mvchte.
%% \AtBeginSection[]
%% {
%%   \begin{frame}<beamer>
%%     \frametitle{Gliederung}
%%     \tableofcontents[currentsection,currentsubsection]
%%   \end{frame}
%% }

% Falls Aufzdhlungen immer schrittweise gezeigt werden sollen, kann
% folgendes Kommando benutzt werden:

\beamerdefaultoverlayspecification{<+->}

%%%%%%
\definecolor{rot}{rgb}{1,0,0}
\definecolor{gruen}{rgb}{0,1,0}
\definecolor{blau}{rgb}{0,0,1}

%%% number sets
\newcommand{\Z}       {\mathbb{Z} }
\newcommand{\R}       {\mathbb{R} }
\newcommand{\Q}       {\mathbb{Q} }
\newcommand{\N}       {\mathbb{N} }
\newcommand{\spa}     {\text{span}}
\newcommand{\lin}     {\text{span}}
\newcommand{\inter}   {\text{int} }


%%% mathematical stuff
\newcommand{\sosR}    {\sum^2}
\newcommand*{\transpose}[1]  { {#1}^T }
\newcommand*{\rounddown}[1]  {\left\floor #1 \right\rfloor}
\newcommand*{\roundup}[1]    {\left\lceil  #1 \right\rceil}
\newcommand*{\ipart}[1]      {\rounddown{#1}}
\newcommand*{\fpart}[1]      {\mathfrak{f}\left(#1\right)}


\newcommand*{\iepoly}[2]  {z_{#1}\left(#2\right)}
\newcommand*{\redmon}[3]  {r_{#1}^{#2}\left( #3 \right)}
\newcommand*{\redset}[1]  {#1^{\emph{red}}}

\newcommand*{\Gpoly}[1] {P_{[#1]}}

\newcommand*{\nonc}[1]{\overline{#1}}
\newcommand*{\const}[1]{#1_0}

\newcommand{\define}{\stackrel{\rm def.}{\Leftrightarrow}}
\newcommand{\qform}[3]{\frac{1}{2} x^{\top}#1x + #2^{\top}x + #3}
\def\xzero{x^{0}}
\newcommand{\gxh}[2]{{g_{#1}(#2)}}

\def\pcone_k{{\mathcal C}_{k}(f)}
\def\orthant_j{{\mathcal O}_{j}}

\def\BDB{BDB^{\top}}
\def\LDL{LDL^{\top}}
\def\bA{A}
\def\bb{b}
\def\bc{c}
\def\bh{h}
\def\bp{p}
\def\bx{x}
\def\by{y}
\def\bu{u}
\def\bv{v}
\def\bd{d}
\def\T{^{T}}
\def\D{}
\def\mb{{\bf}}
\def\sep{}
\def\bo{0}
\def\bw{w}
\def\ba{a}
\def\bg{g}
\def\bH{H}
\def\be{e}

\let\ve=\relax
\newcommand\vealpha{{\alpha}}
\newcommand{\st}{\mathrm{s.t.}}
\DeclareMathOperator\conv{conv}
\DeclareMathOperator\cone{cone}
\newcommand{\B}{\{0,1\}}

\newcommand*{\person}[1] {\textsc{#1}}

\newtheorem{algorithm}{Algorithmus}

\makeatletter
\newenvironment{rmat}{\left(\null\,\vcenter\bgroup
  \Let@\restore@math@cr\default@tag
  \baselineskip6\ex@ \lineskip1.5\ex@ \lineskiplimit\lineskip
  \ialign\bgroup\hfil$\m@th\scriptstyle##$&&\thickspace
  \hfil$\m@th\scriptstyle##$\crcr
}{%
  \crcr\egroup\egroup\,\right)%
}
\makeatother

%%%%%%%%%%%%%%%%%%%%%%%%%%%%%%%%%%%%%%%%%%%%%%%%%%%%%%%%%%%%%%%%%%%%%%%%%%%%%%%%
%\begin{frame}
  %\titlepage
%\end{frame}

%% \begin{frame}
%%   \frametitle{Gliederung}
%%   \tableofcontents
%%   % Die Option [pausesections] kvnnte n|tzlich sein.
%% \end{frame}


%%%%%%%%%%%%%%%%%%%%%%%%%%%%%%%%%%%%%%%%%%%%%%%%%%%%%%%%%%%%%%%%%%

\definecolor{orange}{rgb}{0.8,0.3,0.0}
\definecolor{darkgreen}{rgb}{0.0,0.5,0.0}
\definecolor{gold}{rgb}{1.0,0.8,0.0}
\definecolor{brown}{rgb}{0.6,0.2,0.2}
\definecolor{blue4}{rgb}{0,0,144}
\definecolor{white}{rgb}{255,255,255}
\definecolor{blueexample}{rgb}{0.2,0.2,0.7}
\begin{document}
\begin{frame}
  \titlepage
\end{frame}
\begin{frame}
\frametitle{Valid inequalities}
We have seen that having a good formulation is \alert{crucial}
to obtain a (fast)-solving problem.\\
Main issue: how to automatically \alert{improve} a formulation.
\begin{block}{Definition}
Let $P\subseteq \mathbb R^n$. An inequality $\sum_{j=1}^n a_j x_j \leq b$
is \alert{valid} if it is satisfied by all points $x\in P$.
\end{block}
Typically, we want to derive valid inequalities for the set of
\alert{integral solutions} and at the same time \alert{cut off}
some part of the \alert{linear programming relaxation}.
\end{frame}
\begin{frame}
\frametitle{The rounding principle}
Let $P=\{x\in \mathbb Z^n \mid Ax\leq b\}$ and $P_{LP}=\{x\in \mathbb R^n \mid Ax\leq b\}$
be the corresponding linear programming relaxation.\bigskip

If $x\leq c$ is valid for \alert{$P_{LP}$} then $x\leq \lfloor c\rfloor$ is valid
for \alert{$P$}.
\begin{block}{The Chvatal-Gomory procedure}
\begin{itemize}
\item Compute a nonnegative \alert{combination} of the rows of the LP formulation
$$(u^T A) x \leq u^T b, \qquad (u\geq 0)$$
\item The inequality 
$$(\lfloor u^T A\rfloor ) x \leq \lfloor u^T b\rfloor$$
is valid for $P$.
\end{itemize}
\end{block}
\end{frame}
\begin{frame}
\frametitle{Gomory's fractional cutting plane algorithm}
  \begin{itemize}
  \item Based on the simplex algorithm applied to the linear relaxation of the MIP
  \item automatically generate and apply cuts until solution is integer
    \begin{itemize}
    \item if optimal solution is fractional, use the information provided by the optimal basis to generate cuts (apply the Chvatal-Gomory procedure)
    \end{itemize}
    \item terminates in a finite number of iterations if combined with the right \alert{simplex pivoting rule}
    \item not very successful in practice, hence \textbf{branch-and-cut}.
  \end{itemize}
\end{frame}
 \begin{frame} \frametitle{The Basic Mixed Integer inequality}

   \begin{block} {2D case}
     Let $X = \{ (x,y) \in \mathbb{R}_+ \times \mathbb{Z}_+ \ | \ x + y \geq b\}$ and $f= b - \lfloor b \rfloor > 0$. \\
     Then $$\frac{x}{f} + y \geq \lceil b \rceil$$ is valid for $X$
   \end{block}
  
      \begin{block} {Corollary}
     Let $X = \{ (x,y) \in \mathbb{R}_+ \times \mathbb{Z}_+ \ | \ y \leq b + x\}$ and $f= b - \lfloor b \rfloor > 0$. \\
     Then $$y \leq \lfloor b \rfloor + \frac{x}{1-f}$$ is valid for $X$
   \end{block}

 \end{frame}

 \begin{frame} \frametitle{Mixed Integer Rounding (MIR) cut}
      \begin{block} {}
     Let $$X = \{ (x,y) \in \mathbb{R}_+ \times \mathbb{Z}^2_+ \ | \ a_1y_1 + a_2y_2 \leq b + x\},$$  $$f= b - \lfloor b \rfloor > 0,$$ and $$f_i= a_1 - \lfloor a_i \rfloor, \ i=1,2$$ with $$f_1 \leq f \leq f_2.$$  \\
     Then $$\lfloor a_1 \rfloor y_1 + \left(\lfloor a_2 \rfloor + \frac{f_2-f}{1-f} \right) y_2 \leq \lfloor b \rfloor + \frac{x}{1-f}$$ is valid for $X$.
   \end{block}
 \end{frame}

 \begin{frame} \frametitle{Superadditivity: preliminary definitions}
   \begin{block} {Superadditive function}
     The function $F : D \subseteq \mathbb{R}^m \mapsto \mathbb{R}$ is superadditive if 
     $$ F(a_1) + F(a_2) \leq F(a_1+a_2)$$
     for all $a_1, a_2 \in D$ such that $a_1+a_2 \in D$.
   \end{block}
   Remark: $F$ superadditive $\Rightarrow F(0) \leq 0$.

   \begin{block} {Non-decreasing function}
     The function $F : D \subseteq \mathbb{R}^m \mapsto \mathbb{R}$ is non-decreasing if 
     $$ F(a_1) \leq F(a_2)$$ for all $a_1, a_2 \in D$ such that $a_1 \leq a_2$.
   \end{block}
 \end{frame}

  \begin{frame} \frametitle{Superadditivity}
   \begin{block} {}
     If $F : \mathbb{R}^m \mapsto \mathbb{R}$ is superadditive, non-decreasing and satisfies $F(0) = 0$, then the inequality 
      $$ \sum_{j=1}^n F(A_j)x_j \leq F(b)$$ is valid for $\text{conv(P)}$ with $P = \{x\in \mathbb{Z}^n_+ | Ax \leq b\}$.
   \end{block}

   Proof, comparison to MIR

 \end{frame}


\begin{frame}
\frametitle{Strong inequalities}
\begin{itemize}
\item<1-> Inequalities $\pi x\leq \pi_0$ and $\lambda \pi x\leq \lambda \pi_0$
are \alert{identical} if $\lambda>0$.\bigskip
\item<1-> An inequality $\pi x \leq \pi_0$ \alert{dominates}
$\mu x\leq \mu_0$ if there exists $u>0$ with
$$ \pi\geq u\mu\quad \text{and}\quad \pi_0\leq u\mu_0$$
if we work in a polyhedron $P\subset \mathbb R^n_{\alert{+}}.$
\end{itemize}
\end{frame}
\begin{frame}
\frametitle{Polyhedra, faces and facets}
\begin{itemize}
\item $n$ points $x^{(1)},\ldots, x^{(k)}$ are \alert{affinely independent} 
if $x^{(2)}-x^{(1)},\ldots, x^{(k)}-x^{(1)}$ are \alert{linearly independent}
or equivalently if $(x^{(1)},1),\ldots,(x^{(k)},1)$ are \alert{linearly independent}.
\bigskip
\item The \alert{dimension} $d$ of a polyhedron $P$ is the maximum  number
of affinely independent points  in $P$ \alert{minus 1}.
\bigskip
\item $F$ is a \alert{face} of $P$ if $F=\{x\in P: \pi x=\pi_0\}$
for some valid inequality $\pi x\leq \pi_0$.
\bigskip
\item $F$ is a \alert{facet} if $F$ is a face of $P$ of dimension \alert{dim$(P) -1$}.
\end{itemize}
Facets of conv$(P)$ are the valid inequalities that we are looking for!
\end{frame}
\begin{frame}
\frametitle{Knapsack covers}
We consider the knapsack set
$$X=\{x\in \{0,1\}^n\mid \sum_{j=1}^n a_j x_j\leq b\}.$$
\begin{block}{Definition}
A set $C$ is a \alert{cover} if $\sum_{j\in C} a_j > b$.
\end{block}
\begin{block}{A cover inequality}
If $C$ is a cover, the cover inequality
$$\sum_{j\in C} x_j \leq |C|-1$$
is valid for $X$.
\end{block}
\end{frame}
\begin{frame}
\frametitle{Lifting a cover inequality}
Consider an inequality $\sum_{j\in C} x_j \leq |C|-1.$\\
Consider a \alert{variable} $i\not\in C$ that we would like to \alert{lift},
namely we want to give it a coefficient in the \alert{cover inequality}.
\bigskip

\begin{align*}
\alpha_i = |C|-1 - \max\; & \sum_{j\in C} x_j\\
\text{s. t. }\; & \sum_{j\in C} \leq b - a_i\\
& x_j\in \{0,1\}.
\end{align*}
\end{frame}
\begin{frame} \frametitle{Branch-and-cut: used in all MIP solvers nowadays}
  \begin{itemize}
  \item Branch-and-bound combined with cutting plane algorithm
  \item uses several families of cuts, depending on the problem (MIR, covers, ...)
  \item typically starts as a cutting plane algorithm, then branches
  \item at each node, decide to branch or to generate and add cuts
  \item cuts are often node specific, i.e. cannot be imported in other parts of the tree without care.
  \end{itemize}
\end{frame}

\end{document}
